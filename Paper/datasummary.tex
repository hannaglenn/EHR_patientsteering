\documentclass[12pt]{article}

\usepackage{tgtermes}
\usepackage{epsf}
\usepackage{epstopdf}
\usepackage{amsmath}
\usepackage{graphicx}
\usepackage{booktabs}
\usepackage[colorlinks=true,linkcolor=blue,citecolor=blue]{hyperref}
\usepackage{dcolumn}
\usepackage{amsmath, amsthm, amssymb}
\usepackage{mwe}
\usepackage{url}
%\usepackage{harvard}
\usepackage{fancyheadings}
\usepackage{longtable}
\usepackage{authblk}
\usepackage{setspace}
%\usepackage[nomarkers]{endfloat}
\usepackage{float}
\usepackage{bbm}
%\usepackage{titling}
\usepackage{subcaption}
\usepackage{algorithm}
\usepackage{algorithmic}
\usepackage{import}
\usepackage{colortbl}
\usepackage{color}
%\usepackage[nomarkers,nofiglist,notablist]{endfloat}

\onehalfspacing
\textwidth 6.5in \oddsidemargin 0in \evensidemargin -0.6in
\textheight 8.5in \topmargin -0.2in

\newcolumntype{L}[1]{>{\raggedright\let\newline\\
		\arraybackslash\hspace{0pt}}m{#1}}
\newcolumntype{C}[1]{>{\centering\let\newline\\
		\arraybackslash\hspace{0pt}}m{#1}}
\newcolumntype{R}[1]{>{\raggedleft\let\newline\\
		\arraybackslash\hspace{0pt}}m{#1}}
\newcolumntype{P}[1]{>{\raggedright\tabularxbackslash}p{#1}}

\newtheorem{theorem}{Theorem}[section]
\newtheorem{corollary}[theorem]{Corollary}
\newtheorem{proposition}[theorem]{Proposition}
\newtheorem{lemma}[theorem]{Lemma}

\usepackage[backend=biber,style=authoryear,
sorting=ynt,citestyle=authoryear]{biblatex}
\addbibresource{cites.bib}

\newcommand{\xsub}[1]{%
	\mbox{\scriptsize\begin{tabular}{@{}c@{}}#1\end{tabular}}%
}

%\renewcommand{\thetable}{\Roman{table}}

\begin{document}

\singlespacing

\section{Hospital EHR Info}
Sources:
\begin{itemize}
    \item Healthcare Information and Management Systems Society (HIMSS)
    \begin{itemize}
        \item 1558 hospitals have an EHR in this data
    \end{itemize}
    \item Meaningful use subsidy data from CMS \& CMS IT product information (Health IT Product List)
    \begin{itemize}
        \item 3456 hospitals have an EHR in this data
    \end{itemize}
    \item AHA main survey and IT survey
    \begin{itemize}
        \item 6253 hospitals total
        \item 4542 who ever answer EHR vendor
    \end{itemize}
\end{itemize}

\noindent Since the AHA main survey is the most comprehensive source of hospital characteristics, I join the other sources to this survey by Medicare number. 

\noindent How I define EHR vendor for hospital:
\begin{itemize}
    \item If the hospital is in the meaningful use data, I use this for EHR vendor (since it is administrative)
    \item If the hospital is not in the meaningful use data and answers only one of the surveys, I fill in EHR vendor with the survey answer.
    \item If the hospital is not in the meaningful use data a and answers both of the surveys, I compare vendors in he two surveys. If they match, this is the assigned EHR vendor. If they do not match, I drop the observation. (Only 5\% of observations get dropped here)
    \item If the hospital is not in any of the three data sources, I assume they do not have an EHR. 
\end{itemize}

\noindent Resulting Data:
\begin{itemize}
    \item 33,203 observations: 5,177 hospitals over time
    \item location HRR, system ID, ownership type, EHR vendor
\end{itemize}

\section{PCP specialist pairs}
Sources:
\begin{itemize}
    \item NPPES tax identifications
    \begin{itemize}
        \item limit to PCP (internal medicine, hospitalist, family medicine, general practice), and specialist (contains surgery, specialist, phlebology, otolaryngology, furology, podiatrist)
        \item 832,825 physicians, 69\% PCP
    \end{itemize}
    \item MDPPAS
    \begin{itemize}
        \item only used to limit to physicians who never change tax ID
        \item left with 140,000 physicians after this filtering
    \end{itemize}
    \item Physician Compare
    \begin{itemize}
        \item Used to gather hospital affiliations of each physician
        \item 32\% of PC are dropped for having no hospital affiliation
        \item 28\% of those left in PC  do not have an HRR code associated with the hospital (they were dropped in hospital data creation)
        \item join to NPPES physicians: left with 110,000 phys. after this filtering
    \end{itemize}
    \item CMS Shared Patient Data
\end{itemize}

\noindent Creating Pairs:
\begin{itemize}
    \item I have 53,000 PCPs and 57,000 specialists. 
    \item Limit the specialists to having a sufficient total number of specialists available (over 100). left with 30 specialties
    \item For every PCP, I join ach specialist located in the same HRR, creating a data set where each row is a PCP-specialist pair in the same HRR, which is balanced from 2010-2014
    \item Join the CMS shared patient data to these pairs, filling in with zero if the pair is not present in this data
    \item Limit to PCPs who send positive referrals in at least two years of the data
    \item Join hospital level information based on affiliated hospital
\end{itemize}

\section{PCP specialist practice pairs}

\begin{itemize}
    \item Start with pairs created in previous section. Aggregate over practice affiliation by summing shared patients for all the PCPs or specialists in a practice
    \item Based on how many shared patients each physician has with hospitals, I form each practice's affiliated hospitals. 
    \item For example, consider a practice that consists of PCP A and PCP B, where PCP A shares 5 patients with Hospital 1 and 10 patients with Hospital 2, and PCP B shares 20 patients with Hospital 1. In total, the practice shares 35 patients with hospitals 1 and 2. 25 of those patients are shared with Hospital1, and the remaining 10 are shared with Hospital 2. Thus, Hospital 1 is the practice's main affiliated hospital, and hospital 2 is the practice's secondary affiliated hospital.
    \item The resulting data consists of PCP practice NPI, specialist practice NPI, and characteristics of the affiliated hospitals of each practice.  
\end{itemize}
	
	
\section{Graphs}
-> just ortho. surgeons

effect of PCP's main hospital switching on fraction of patients sent to specialist with same main hospital affiliation:


\end{document}